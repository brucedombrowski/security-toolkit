\documentclass[11pt,letterpaper]{article}

% ============================================================================
% PACKAGES
% ============================================================================
\usepackage[margin=1in, top=1.5in, headheight=85pt]{geometry}
\usepackage{graphicx}
\usepackage{fancyhdr}
\usepackage{lastpage}
\usepackage{enumitem}
\usepackage{xcolor}
\usepackage{hyperref}
\usepackage{tabularx}
\usepackage{booktabs}
\usepackage{listings}
\usepackage{longtable}

% ============================================================================
% DOCUMENT VARIABLES - These are replaced by the generation script
% ============================================================================
\newcommand{\UniqueID}{MAL-YYYY-NNN}
\newcommand{\DocumentDate}{January 15, 2026}
\newcommand{\ScanTimestamp}{2026-01-15 08:00:00}
\newcommand{\TargetName}{ProjectName}
\newcommand{\ScanScope}{Local Scan}
\newcommand{\ScanTarget}{/path/to/target}
\newcommand{\SecurityToolkitVersion}{v1.0.0}
\newcommand{\SecurityToolkitCommit}{abc1234}
\newcommand{\SecurityToolkitURL}{https://github.com/brucedombrowski/Security}

% ClamAV Scanner Info
\newcommand{\ClamAVVersion}{ClamAV 1.0.0}
\newcommand{\VirusDBVersion}{27000}
\newcommand{\VirusDBDate}{Jan 01, 2026}
\newcommand{\SignatureCount}{3000000}

% Scan Results
\newcommand{\ScanResult}{PASS}
\newcommand{\InfectedCount}{0}
\newcommand{\ScannedFilesCount}{100}
\newcommand{\ScannedDirsCount}{10}
\newcommand{\DataScanned}{10.5 MiB}
\newcommand{\ScanDuration}{5.2 sec}

% Source File Reference
\newcommand{\SourceFileName}{malware-scan-2026-01-15.txt}
\newcommand{\SourceFileChecksum}{0000000000000000000000000000000000000000000000000000000000000000}
\newcommand{\ClamAVLogChecksum}{0000000000000000000000000000000000000000000000000000000000000000}

% Scan Machine (host running the toolkit)
\newcommand{\ScanMachineID}{N/A}
\newcommand{\ScanMachineIDType}{Fingerprint}
\newcommand{\ScanMachineIDMethod}{SHA256(MAC:hostname)}

% Target Machine (host being scanned - same as scan machine for local scans)
\newcommand{\TargetMachineID}{N/A}
\newcommand{\TargetMachineIDType}{Fingerprint}
\newcommand{\TargetMachineIDMethod}{SHA256(MAC:hostname)}
\newcommand{\TargetInventoryFile}{N/A}

% NIST Control
\newcommand{\NISTControl}{SI-3}
\newcommand{\NISTControlName}{Malicious Code Protection}

% ============================================================================
% DOCUMENT CONFIGURATION
% ============================================================================
\hypersetup{
    colorlinks=true,
    linkcolor=blue,
    urlcolor=blue,
    pdfborder={0 0 0}
}

\setlength{\parindent}{0pt}
\setlength{\parskip}{6pt}

\setlist[enumerate]{leftmargin=0.5in, labelwidth=0.25in, labelsep=0.1in, align=left, itemsep=12pt, topsep=12pt}
\setlist[enumerate,1]{label=\arabic*.}
\setlist[itemize]{leftmargin=0.5in, itemsep=3pt, topsep=6pt}

% Code listing colors
\definecolor{codebg}{rgb}{0.95,0.95,0.95}
\definecolor{codegray}{rgb}{0.5,0.5,0.5}

% Configure listings for hash display
\lstset{
    backgroundcolor=\color{codebg},
    basicstyle=\ttfamily\footnotesize,
    breaklines=true,
    columns=fullflexible,
    keepspaces=true,
    numbers=left,
    numberstyle=\tiny\color{codegray},
    numbersep=5pt,
    frame=single,
    framesep=5pt,
    xleftmargin=15pt,
    xrightmargin=5pt
}

% ============================================================================
% HEADER AND FOOTER CONFIGURATION
% ============================================================================
\pagestyle{fancy}
\fancyhf{}

\fancyhead[L]{%
    \raisebox{-0.5\height}{\includegraphics[height=0.9in]{logo.png}}%
}
\fancyhead[R]{%
    \parbox[b]{3.5in}{\raggedleft\large\textbf{Malware Scan Attestation}}%
}
\renewcommand{\headrulewidth}{0pt}

\fancyfoot[L]{\small \UniqueID}
\fancyfoot[C]{\small Page \thepage\ of \pageref{LastPage}}
\fancyfoot[R]{\small \DocumentDate}
\renewcommand{\footrulewidth}{0.4pt}

% ============================================================================
% CUSTOM COMMANDS
% ============================================================================
\newcommand{\dmsection}[1]{\item \textbf{#1}\par}

% Result formatting
\newcommand{\passmark}{\textcolor{green!60!black}{\textbf{PASS}}}
\newcommand{\failmark}{\textcolor{red}{\textbf{FAIL}}}
\newcommand{\skipmark}{\textcolor{gray}{\textbf{SKIP}}}
\newcommand{\resultcell}[1]{\ifnum\pdfstrcmp{#1}{PASS}=0 \passmark\else\ifnum\pdfstrcmp{#1}{SKIP}=0 \skipmark\else\failmark\fi\fi}

% ============================================================================
% BEGIN DOCUMENT
% ============================================================================
\begin{document}

\hfill \UniqueID

\vspace{0.25in}

% ----------------------------------------------------------------------------
% HEADER FIELDS
% ----------------------------------------------------------------------------
\textbf{DATE:} \DocumentDate

\textbf{SUBJECT:} ClamAV Malware Scan Attestation (NIST \NISTControl)

\vspace{0.25in}

% ============================================================================
% MAIN CONTENT
% ============================================================================
\begin{enumerate}

% ----------------------------------------------------------------------------
\dmsection{Purpose}

This document attests that a ClamAV malware scan was executed against the specified target. This scan verifies alignment with NIST SP 800-53 control \NISTControl{} (\NISTControlName).

% ----------------------------------------------------------------------------
\dmsection{Scan Environment}

\begin{itemize}
    \item \textbf{Scan Timestamp:} \ScanTimestamp
    \item \textbf{Target Path:} \texttt{\ScanTarget}
    \item \textbf{Toolkit Version:} \SecurityToolkitVersion{} (\SecurityToolkitCommit)
    \item \textbf{Toolkit Repository:} \url{\SecurityToolkitURL}
\end{itemize}

% ----------------------------------------------------------------------------
\dmsection{Machine Identification}

\textbf{3.1 Scan Machine} (host running the security toolkit)

\vspace{6pt}
\begin{tabular}{@{}ll@{}}
\toprule
\textbf{Identifier} & {\small\texttt{\ScanMachineID}} \\
\textbf{Type} & \ScanMachineIDType \\
\textbf{Method} & \ScanMachineIDMethod \\
\bottomrule
\end{tabular}

\vspace{12pt}
\textbf{3.2 Target Machine} (host being scanned)

\vspace{6pt}
\begin{tabular}{@{}ll@{}}
\toprule
\textbf{Identifier} & {\small\texttt{\TargetMachineID}} \\
\textbf{Type} & \TargetMachineIDType \\
\textbf{Inventory File} & \texttt{\TargetInventoryFile} \\
\textbf{Method} & \TargetMachineIDMethod \\
\bottomrule
\end{tabular}

\vspace{6pt}
{\small\textit{Note: Machine identifiers are derived from hardware attributes (MAC address, hostname) to enable verification while protecting raw values. For local scans, Scan Machine and Target Machine are identical.}}

% ----------------------------------------------------------------------------
\dmsection{Scanner Information}

\vspace{6pt}
\begin{tabular}{@{}ll@{}}
\toprule
\textbf{Property} & \textbf{Value} \\
\midrule
Scanner & \ClamAVVersion \\
Virus DB Version & \VirusDBVersion \\
Virus DB Date & \VirusDBDate \\
Signature Count & \SignatureCount \\
\bottomrule
\end{tabular}

% ----------------------------------------------------------------------------
\dmsection{Scan Results}

\vspace{6pt}
\begin{tabular}{@{}ll@{}}
\toprule
\textbf{Metric} & \textbf{Value} \\
\midrule
Result & \resultcell{\ScanResult} \\
Infected Files & \InfectedCount \\
Files Scanned & \ScannedFilesCount \\
Directories Scanned & \ScannedDirsCount \\
Data Scanned & \DataScanned \\
Scan Duration & \ScanDuration \\
\bottomrule
\end{tabular}

\vspace{12pt}
\textbf{Overall Result:} \resultcell{\ScanResult}

% ----------------------------------------------------------------------------
\dmsection{Source File Reference}

This attestation is derived from the following scan output files. The SHA256 checksums can be used to verify file integrity.

\vspace{6pt}
\begin{tabular}{@{}ll@{}}
\toprule
\textbf{File} & \texttt{\SourceFileName} \\
\textbf{SHA256} & {\scriptsize\texttt{\SourceFileChecksum}} \\
\bottomrule
\end{tabular}

\vspace{6pt}
\textbf{Verification Command:}
\begin{verbatim}
shasum -a 256 .scans/<filename>
\end{verbatim}

% ----------------------------------------------------------------------------
\dmsection{NIST Control Mapping}

\vspace{6pt}
\begin{tabular}{@{}llp{3in}@{}}
\toprule
\textbf{Control} & \textbf{Family} & \textbf{Description} \\
\midrule
SI-3 & System \& Info Integrity & Malicious Code Protection \\
SI-3(1) & System \& Info Integrity & Central Management \\
SI-3(2) & System \& Info Integrity & Automatic Updates \\
\bottomrule
\end{tabular}

\vspace{6pt}
\textbf{Control Implementation:}
\begin{itemize}
    \item ClamAV provides signature-based malware detection
    \item Virus definitions are regularly updated via freshclam
    \item Scan results are logged with file-level detail
    \item File hashes (SHA256, MD5, SHA1) are recorded for forensic analysis
\end{itemize}

% ----------------------------------------------------------------------------
\dmsection{Attestation}

This document certifies that:

\begin{enumerate}[label=(\alph*)]
    \item A ClamAV malware scan was executed against the specified target
    \item Scan results are accurately represented in this document
    \item The source file checksums enable independent verification
    \item Detailed file hashes are available in Appendix A
\end{enumerate}

\end{enumerate}

% ============================================================================
% APPENDIX A - Scanned Files with Hashes
% ============================================================================
\newpage
\section*{Appendix A: Scanned Files with Hashes}

The following table lists all files scanned with their cryptographic hashes. These hashes can be used for:
\begin{itemize}
    \item Verifying file integrity over time
    \item Identifying duplicate files across systems
    \item Cross-referencing with threat intelligence databases
    \item Forensic analysis and chain of custody
\end{itemize}

\vspace{12pt}
\input{file_hashes.tex}

\end{document}

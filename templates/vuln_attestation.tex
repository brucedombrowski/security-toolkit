\documentclass[11pt,letterpaper]{article}

% ============================================================================
% PACKAGES
% ============================================================================
\usepackage[margin=1in, top=1.5in, headheight=85pt]{geometry}
\usepackage{graphicx}
\usepackage{fancyhdr}
\usepackage{lastpage}
\usepackage{enumitem}
\usepackage{xcolor}
\usepackage{hyperref}
\usepackage{tabularx}
\usepackage{booktabs}
\usepackage{listings}
\usepackage{longtable}

% ============================================================================
% DOCUMENT VARIABLES - Replaced by generation script
% ============================================================================
\newcommand{\UniqueID}{VULN-YYYY-NNN}
\newcommand{\DocumentDate}{January 15, 2026}
\newcommand{\ScanTimestamp}{2026-01-15 08:00:00}
\newcommand{\SecurityToolkitVersion}{v1.0.0}
\newcommand{\SecurityToolkitCommit}{abc1234}
\newcommand{\SecurityToolkitURL}{https://github.com/brucedombrowski/Security}

% Scanner Info
\newcommand{\ScannerName}{Lynis}
\newcommand{\ScannerVersion}{3.0.0}
\newcommand{\ScanMode}{normal}
\newcommand{\ScanTarget}{(local) /}

% Scan Machine (host running the toolkit)
\newcommand{\ScanMachineID}{N/A}
\newcommand{\ScanMachineIDType}{Fingerprint}
\newcommand{\ScanMachineIDMethod}{SHA256(MAC:hostname)}

% Target Machine (host being scanned)
\newcommand{\TargetMachineID}{N/A}
\newcommand{\TargetMachineIDType}{Fingerprint}
\newcommand{\TargetMachineIDMethod}{SHA256(MAC:hostname)}
\newcommand{\TargetInventoryFile}{N/A}

% Scan Results
\newcommand{\ScanResult}{PASS}
\newcommand{\HardeningIndex}{0}
\newcommand{\WarningsCount}{0}
\newcommand{\SuggestionsCount}{0}
\newcommand{\TestsPerformed}{0}
\newcommand{\PluginsEnabled}{0}

% Source File Reference
\newcommand{\SourceFileName}{lynis-scan-2026-01-15.txt}
\newcommand{\SourceFileChecksum}{0000000000000000000000000000000000000000000000000000000000000000}

% NIST Controls
\newcommand{\NISTControls}{RA-5, SI-2, CM-6}

% ============================================================================
% DOCUMENT CONFIGURATION
% ============================================================================
\hypersetup{
    colorlinks=true,
    linkcolor=blue,
    urlcolor=blue,
    pdfborder={0 0 0}
}

\setlength{\parindent}{0pt}
\setlength{\parskip}{6pt}

\setlist[enumerate]{leftmargin=0.5in, labelwidth=0.25in, labelsep=0.1in, align=left, itemsep=12pt, topsep=12pt}
\setlist[enumerate,1]{label=\arabic*.}
\setlist[itemize]{leftmargin=0.5in, itemsep=3pt, topsep=6pt}

% Code listing colors
\definecolor{codebg}{rgb}{0.95,0.95,0.95}
\definecolor{codegray}{rgb}{0.5,0.5,0.5}

% Lynis output colors
\definecolor{lynisgreen}{rgb}{0.0,0.6,0.0}
\definecolor{lynisyellow}{rgb}{0.8,0.6,0.0}
\definecolor{lynisred}{rgb}{0.8,0.0,0.0}
\definecolor{lyniscyan}{rgb}{0.0,0.5,0.7}
\definecolor{lyniswhite}{rgb}{0.3,0.3,0.3}

\lstset{
    backgroundcolor=\color{codebg},
    basicstyle=\ttfamily\scriptsize,
    breaklines=true,
    columns=fullflexible,
    keepspaces=true,
    numbers=left,
    numberstyle=\tiny\color{codegray},
    numbersep=5pt,
    frame=single,
    framesep=5pt,
    xleftmargin=15pt,
    xrightmargin=5pt
}

% Colored status commands for Lynis output
\newcommand{\lynisOK}{\textcolor{lynisgreen}{[ OK ]}}
\newcommand{\lynisDONE}{\textcolor{lynisgreen}{[ DONE ]}}
\newcommand{\lynisFOUND}{\textcolor{lynisgreen}{[ FOUND ]}}
\newcommand{\lynisWARNING}{\textcolor{lynisyellow}{[ WARNING ]}}
\newcommand{\lynisSUGGESTION}{\textcolor{lynisyellow}{[ SUGGESTION ]}}
\newcommand{\lynisNOTFOUND}{\textcolor{codegray}{[ NOT FOUND ]}}  % Gray - context-dependent
\newcommand{\lynisNONE}{\textcolor{lynisred}{[ NONE ]}}
\newcommand{\lynisFAIL}{\textcolor{lynisred}{[ FAIL ]}}
\newcommand{\lynisUNKNOWN}{\textcolor{lynisred}{[ UNKNOWN ]}}
\newcommand{\lynisSKIPPED}{\textcolor{codegray}{[ SKIPPED ]}}

% ============================================================================
% HEADER AND FOOTER
% ============================================================================
\pagestyle{fancy}
\fancyhf{}

\fancyhead[L]{%
    \raisebox{-0.5\height}{\includegraphics[height=0.9in]{logo.png}}%
}
\fancyhead[R]{%
    \parbox[b]{3.5in}{\raggedleft\large\textbf{Vulnerability Scan Attestation}}%
}
\renewcommand{\headrulewidth}{0pt}

\fancyfoot[L]{\small \UniqueID}
\fancyfoot[C]{\small Page \thepage\ of \pageref{LastPage}}
\fancyfoot[R]{\small \DocumentDate}
\renewcommand{\footrulewidth}{0.4pt}

% ============================================================================
% CUSTOM COMMANDS
% ============================================================================
\newcommand{\dmsection}[1]{\item \textbf{#1}\par}

\newcommand{\passmark}{\textcolor{green!60!black}{\textbf{PASS}}}
\newcommand{\failmark}{\textcolor{red}{\textbf{FAIL}}}
\newcommand{\skipmark}{\textcolor{gray}{\textbf{SKIP}}}
\newcommand{\resultcell}[1]{\ifnum\pdfstrcmp{#1}{PASS}=0 \passmark\else\ifnum\pdfstrcmp{#1}{SKIP}=0 \skipmark\else\failmark\fi\fi}

% ============================================================================
% BEGIN DOCUMENT
% ============================================================================
\begin{document}

\hfill \UniqueID

\vspace{0.25in}

% ----------------------------------------------------------------------------
% HEADER FIELDS
% ----------------------------------------------------------------------------
\textbf{DATE:} \DocumentDate

\textbf{SUBJECT:} Vulnerability Scan Attestation (\ScannerName) - NIST \NISTControls

\vspace{0.25in}

% ============================================================================
% MAIN CONTENT
% ============================================================================
\begin{enumerate}

% ----------------------------------------------------------------------------
\dmsection{Purpose}

This document attests that a vulnerability assessment was performed using \ScannerName{} security auditing tool. This scan verifies alignment with NIST SP 800-53 controls \NISTControls{} (Vulnerability Monitoring, Flaw Remediation, Configuration Settings).

% ----------------------------------------------------------------------------
\dmsection{Scan Environment}

\begin{itemize}
    \item \textbf{Scan Timestamp:} \ScanTimestamp
    \item \textbf{Target Path:} \texttt{\ScanTarget}
    \item \textbf{Scan Mode:} \ScanMode
    \item \textbf{Toolkit Version:} \SecurityToolkitVersion{} (\SecurityToolkitCommit)
    \item \textbf{Toolkit Repository:} \url{\SecurityToolkitURL}
\end{itemize}

% ----------------------------------------------------------------------------
\dmsection{Machine Identification}

\textbf{3.1 Scan Machine} (host running the security toolkit)

\vspace{6pt}
\begin{tabular}{@{}ll@{}}
\toprule
\textbf{Identifier} & {\small\texttt{\ScanMachineID}} \\
\textbf{Type} & \ScanMachineIDType \\
\textbf{Method} & \ScanMachineIDMethod \\
\bottomrule
\end{tabular}

\vspace{12pt}
\textbf{3.2 Target Machine} (host being scanned)

\vspace{6pt}
\begin{tabular}{@{}ll@{}}
\toprule
\textbf{Identifier} & {\small\texttt{\TargetMachineID}} \\
\textbf{Type} & \TargetMachineIDType \\
\textbf{Inventory File} & \texttt{\TargetInventoryFile} \\
\textbf{Method} & \TargetMachineIDMethod \\
\bottomrule
\end{tabular}

\vspace{6pt}
{\small\textit{Note: For local scans, Scan Machine and Target Machine are identical.}}

% ----------------------------------------------------------------------------
\dmsection{Scanner Information}

\vspace{6pt}
\begin{tabular}{@{}ll@{}}
\toprule
\textbf{Property} & \textbf{Value} \\
\midrule
Scanner & \ScannerName \\
Version & \ScannerVersion \\
Scan Mode & \ScanMode \\
Tests Performed & \TestsPerformed \\
Plugins Enabled & \PluginsEnabled \\
\bottomrule
\end{tabular}

% ----------------------------------------------------------------------------
\dmsection{Scan Results}

\vspace{6pt}
\begin{tabular}{@{}ll@{}}
\toprule
\textbf{Metric} & \textbf{Value} \\
\midrule
Overall Result & \resultcell{\ScanResult} \\
Hardening Index & \HardeningIndex \\
Warnings & \WarningsCount \\
Suggestions & \SuggestionsCount \\
\bottomrule
\end{tabular}

\vspace{12pt}
\textbf{Overall Result:} \resultcell{\ScanResult}

\vspace{6pt}
{\small\textit{Note: Hardening index ranges from 0-100. Higher values indicate better security posture. Warnings indicate potential security issues. Suggestions are recommendations for improvement.}}

% ----------------------------------------------------------------------------
\dmsection{Source File Reference}

\vspace{6pt}
\begin{tabular}{@{}ll@{}}
\toprule
\textbf{File} & \texttt{\SourceFileName} \\
\textbf{SHA256} & {\scriptsize\texttt{\SourceFileChecksum}} \\
\bottomrule
\end{tabular}

\vspace{6pt}
\textbf{Verification Command:}
\begin{verbatim}
shasum -a 256 .scans/<filename>
\end{verbatim}

% ----------------------------------------------------------------------------
\dmsection{NIST Control Mapping}

\vspace{6pt}
\begin{tabular}{@{}llp{2.8in}@{}}
\toprule
\textbf{Control} & \textbf{Family} & \textbf{Description} \\
\midrule
RA-5 & Risk Assessment & Vulnerability Monitoring and Scanning \\
SI-2 & System Integrity & Flaw Remediation \\
CM-6 & Config Management & Configuration Settings \\
CA-2 & Security Assessment & Control Assessments \\
\bottomrule
\end{tabular}

% ----------------------------------------------------------------------------
\dmsection{Findings}

The following findings require attention. This is a statement of facts, not an assessment.

\input{vuln_key_findings.tex}

% ----------------------------------------------------------------------------
\dmsection{Attestation}

This document certifies that:

\begin{enumerate}[label=(\alph*)]
    \item A \ScannerName{} vulnerability scan was executed against the specified target
    \item Scan results are accurately represented in this document
    \item The source file checksum enables independent verification
    \item Full scan output is available in Appendix A
\end{enumerate}

\end{enumerate}

% ============================================================================
% APPENDIX A - Full Scan Output
% ============================================================================
\newpage
\section*{Appendix A: Full Scan Output}

The following is the complete \ScannerName{} scan output with color-coded status indicators.

\vspace{6pt}
\textbf{Legend:}
\lynisOK{} = Passed \quad
\lynisDONE{} = Completed \quad
\lynisWARNING{} = Warning \quad
\lynisSUGGESTION{} = Suggestion \quad
\lynisNOTFOUND{} = Not Found \quad
\lynisNONE{} = None

\vspace{12pt}

\input{vuln_findings.tex}

\end{document}

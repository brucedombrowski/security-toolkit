\documentclass[11pt,letterpaper]{article}

% ============================================================================
% PACKAGES
% ============================================================================
\usepackage[margin=1in, top=1.5in, headheight=85pt]{geometry}
\usepackage{graphicx}
\usepackage{fancyhdr}
\usepackage{lastpage}
\usepackage{enumitem}
\usepackage{xcolor}
\usepackage{hyperref}
\usepackage{tabularx}
\usepackage{booktabs}

% ============================================================================
% DOCUMENT VARIABLES - These are replaced by the generation script
% ============================================================================
\newcommand{\UniqueID}{SCAN-YYYY-NNN}
\newcommand{\DocumentDate}{January 15, 2026}
\newcommand{\ScanTimestamp}{2026-01-15 08:00:00}
\newcommand{\TargetHost}{PLACEHOLDER_HOSTNAME}
\newcommand{\TargetPath}{/path/to/project}
\newcommand{\TargetName}{ProjectName}
\newcommand{\SecurityToolkitVersion}{v1.0.0}
\newcommand{\SecurityToolkitCommit}{abc1234}
\newcommand{\SecurityToolkitURL}{https://github.com/brucedombrowski/Security}

% Host inventory reference
\newcommand{\HostInventoryChecksum}{0000000000000000000000000000000000000000000000000000000000000000}
\newcommand{\HostInventoryFile}{host-inventory-2026-01-15.txt}
\newcommand{\ScanDateStamp}{2026-01-15}
\newcommand{\ScanFileTimestamp}{2026-01-15T000000Z}

% Scan results (PASS/FAIL)
\newcommand{\PIIScanResult}{PASS}
\newcommand{\PIIScanFindings}{No PII detected}
\newcommand{\MalwareScanResult}{PASS}
\newcommand{\MalwareScanFindings}{No malware detected}
\newcommand{\SecretsScanResult}{PASS}
\newcommand{\SecretsScanFindings}{No secrets detected}
\newcommand{\MACScanResult}{PASS}
\newcommand{\MACScanFindings}{No MAC addresses detected}
\newcommand{\HostSecurityResult}{PASS}
\newcommand{\HostSecurityFindings}{All checks passed}
\newcommand{\VulnScanResult}{SKIP}
\newcommand{\VulnScanFindings}{Not run}
\newcommand{\VulnScanRan}{false}
\newcommand{\OverallResult}{PASS}
\newcommand{\PassCount}{5}
\newcommand{\FailCount}{0}

% PII Allowlist/Exceptions
\newcommand{\PIIAllowlistCount}{0}
\newcommand{\PIIAllowlistChecksum}{N/A}

% Secrets Allowlist/Exceptions
\newcommand{\SecretsAllowlistCount}{0}
\newcommand{\SecretsAllowlistChecksum}{N/A}

% Scan output checksums (first 16 chars of SHA256)
\newcommand{\PIIScanChecksum}{N/A}
\newcommand{\MalwareScanChecksum}{N/A}
\newcommand{\SecretsScanChecksum}{N/A}
\newcommand{\MACScanChecksum}{N/A}
\newcommand{\HostSecurityScanChecksum}{N/A}
\newcommand{\ReportChecksum}{N/A}
\newcommand{\VulnScanChecksum}{N/A}
\newcommand{\ChecksumsMdChecksumFull}{CHECKSUMS_MD_FULL_PLACEHOLDER}

% ============================================================================
% DOCUMENT CONFIGURATION
% ============================================================================
\hypersetup{
    colorlinks=true,
    linkcolor=blue,
    urlcolor=blue,
    pdfborder={0 0 0}
}

\setlength{\parindent}{0pt}
\setlength{\parskip}{6pt}

\setlist[enumerate]{leftmargin=0.5in, labelwidth=0.25in, labelsep=0.1in, align=left, itemsep=12pt, topsep=12pt}
\setlist[enumerate,1]{label=\arabic*.}
\setlist[itemize]{leftmargin=0.5in, itemsep=3pt, topsep=6pt}

% ============================================================================
% HEADER AND FOOTER CONFIGURATION
% ============================================================================
\pagestyle{fancy}
\fancyhf{}

\fancyhead[L]{%
    \raisebox{-0.5\height}{\includegraphics[height=0.9in]{logo.png}}%
}
\fancyhead[R]{%
    \parbox[b]{3.5in}{\raggedleft\large\textbf{Security Scan Attestation}}%
}
\renewcommand{\headrulewidth}{0pt}

\fancyfoot[L]{\small \UniqueID}
\fancyfoot[C]{\small Page \thepage\ of \pageref{LastPage}}
\fancyfoot[R]{\small \DocumentDate}
\renewcommand{\footrulewidth}{0.4pt}

% ============================================================================
% CUSTOM COMMANDS
% ============================================================================
\newcommand{\dmsection}[1]{\item \textbf{#1}\par}

% Result formatting
\newcommand{\passmark}{\textcolor{green!60!black}{\textbf{PASS}}}
\newcommand{\passexceptmark}{\textcolor{green!60!black}{\textbf{PASS w/ exceptions}}}
\newcommand{\failmark}{\textcolor{red}{\textbf{FAIL}}}
\newcommand{\skipmark}{\textcolor{gray}{\textbf{SKIP}}}
\newcommand{\resultcell}[1]{\ifnum\pdfstrcmp{#1}{PASS}=0 \passmark\else\ifnum\pdfstrcmp{#1}{EXCEPT}=0 \passexceptmark\else\ifnum\pdfstrcmp{#1}{SKIP}=0 \skipmark\else\failmark\fi\fi\fi}

% ============================================================================
% BEGIN DOCUMENT
% ============================================================================
\begin{document}

\hfill \UniqueID

\vspace{0.25in}

% ----------------------------------------------------------------------------
% HEADER FIELDS
% ----------------------------------------------------------------------------
\textbf{DATE:} \DocumentDate

\textbf{HOST:} \texttt{\TargetHost}

\textbf{PATH:} \texttt{\TargetPath} (\TargetName)

\textbf{SUBJECT:} Automated Security Scan Attestation

\vspace{0.25in}

% ============================================================================
% MAIN CONTENT
% ============================================================================
\begin{enumerate}

% ----------------------------------------------------------------------------
\dmsection{Purpose}

This document attests that automated security scans were executed against the subject host at the specified path using the Security Verification Toolkit. The scans verify alignment with NIST SP 800-53 security controls.

% ----------------------------------------------------------------------------
\dmsection{Scan Environment}

\begin{itemize}
    \item \textbf{Scan Timestamp:} \ScanTimestamp
    \item \textbf{Toolkit Version:} \SecurityToolkitVersion{} (\SecurityToolkitCommit)
    \item \textbf{Toolkit Repository:} \url{\SecurityToolkitURL}
\end{itemize}

% ----------------------------------------------------------------------------
\dmsection{Host Inventory Reference}

A host inventory snapshot was collected at scan time to establish a verifiable system thumbprint. This enables integrity verification while keeping sensitive machine data (MAC addresses, serial numbers) separate from shareable scan results.

\vspace{6pt}
\begin{tabular}{@{}ll@{}}
\toprule
\textbf{Inventory File} & \texttt{\HostInventoryFile} \\
\textbf{SHA256 Checksum} & \texttt{\small \HostInventoryChecksum} \\
\textbf{NIST Control} & CM-8 (System Component Inventory) \\
\bottomrule
\end{tabular}

\vspace{6pt}
\textbf{Note:} The host inventory contains sensitive information. All scan outputs reference this checksum rather than embedding the actual data, allowing scan results to be shared without exposing machine-specific details.

% ----------------------------------------------------------------------------
\dmsection{NIST Control Mapping}

The following NIST SP 800-53 Rev 5 controls were verified:

\vspace{6pt}
\begin{tabular}{@{}llp{2.5in}@{}}
\toprule
\textbf{Control} & \textbf{Family} & \textbf{Verification Method} \\
\midrule
CM-6 & Configuration Management & Host security posture \\
CM-8 & Configuration Management & Host inventory collection \\
RA-5 & Risk Assessment & Vulnerability scanning (Nmap/Lynis) \\
SA-11 & System \& Services Acquisition & Secrets and credential scanning \\
SC-8 & System \& Comms Protection & MAC address detection \\
SI-2 & System \& Info Integrity & Flaw remediation assessment \\
SI-3 & System \& Info Integrity & ClamAV malware scanning \\
SI-4 & System \& Info Integrity & System monitoring (port scan) \\
SI-12 & System \& Info Integrity & PII pattern detection \\
\bottomrule
\end{tabular}

% ----------------------------------------------------------------------------
\dmsection{Scan Results}

\vspace{6pt}
\begin{tabular}{@{}llll@{}}
\toprule
\textbf{Scan} & \textbf{NIST Control} & \textbf{Result} & \textbf{Findings} \\
\midrule
Host Security & CM-6 & \resultcell{\HostSecurityResult} & \HostSecurityFindings \\
Vulnerability Scan & RA-5 & \resultcell{\VulnScanResult} & \VulnScanFindings \\
Secrets Scan & SA-11 & \resultcell{\SecretsScanResult} & \SecretsScanFindings \\
MAC Address Scan & SC-8 & \resultcell{\MACScanResult} & \MACScanFindings \\
Malware Scan & SI-3 & \resultcell{\MalwareScanResult} & \MalwareScanFindings \\
PII Scan & SI-12 & \resultcell{\PIIScanResult} & \PIIScanFindings \\
\bottomrule
\end{tabular}

\vspace{12pt}
\textbf{Summary:} \PassCount{} passed, \FailCount{} failed

\vspace{6pt}
\textbf{Overall Result:} \resultcell{\OverallResult}

\vspace{6pt}
{\small PASS w/ exceptions = reviewed exceptions documented in Section 6}

% ----------------------------------------------------------------------------
\dmsection{Reviewed Exceptions}

Items flagged by automated scans but reviewed and accepted as non-issues are documented in allowlist files. Each exception includes a SHA256 hash for integrity verification and reviewer justification.

\vspace{12pt}
\textbf{6.1 PII Scan Exceptions}

\vspace{6pt}
\begin{tabular}{@{}ll@{}}
\toprule
\textbf{Allowlist File} & \texttt{.allowlists/pii-allowlist} \\
\textbf{SHA256 (first 16)} & \texttt{\PIIAllowlistChecksum} \\
\textbf{Total Exceptions} & \PIIAllowlistCount \\
\bottomrule
\end{tabular}

\vspace{6pt}
{\small
\begin{tabular}{@{}cp{4.5in}@{}}
\toprule
\textbf{\#} & \textbf{Justification} \\
\midrule
\input{pii_entries.tex}
\bottomrule
\end{tabular}
}

\vspace{12pt}
\textbf{6.2 Secrets Scan Exceptions}

\vspace{6pt}
\begin{tabular}{@{}ll@{}}
\toprule
\textbf{Allowlist File} & \texttt{.allowlists/secrets-allowlist} \\
\textbf{SHA256 (first 16)} & \texttt{\SecretsAllowlistChecksum} \\
\textbf{Total Exceptions} & \SecretsAllowlistCount \\
\bottomrule
\end{tabular}

\vspace{6pt}
{\small
\begin{tabular}{@{}cp{4.5in}@{}}
\toprule
\textbf{\#} & \textbf{Justification} \\
\midrule
\input{secrets_entries.tex}
\bottomrule
\end{tabular}
}

\vspace{6pt}
{\small\textbf{Note:} Full details including file paths and SHA256 hashes are in the respective allowlist files.}

% ----------------------------------------------------------------------------
\dmsection{Scan Output Checksums}

The following SHA256 checksums were generated for scan output files:

\vspace{6pt}
{\small
\begin{tabular}{@{}ll@{}}
\toprule
\textbf{Scan Output File} & \textbf{SHA256 (first 16)} \\
\midrule
host-inventory & \texttt{\HostInventoryChecksum} \\
host-security-scan & \texttt{\HostSecurityScanChecksum} \\
mac-address-scan & \texttt{\MACScanChecksum} \\
malware-scan & \texttt{\MalwareScanChecksum} \\
pii-scan & \texttt{\PIIScanChecksum} \\
secrets-scan & \texttt{\SecretsScanChecksum} \\
vulnerability-scan & \texttt{\VulnScanChecksum} \\
security-scan-report & \texttt{\ReportChecksum} \\
\bottomrule
\end{tabular}
}

\vspace{12pt}
\textbf{checksums.md SHA256 (full):}

\vspace{3pt}
{\small\texttt{\ChecksumsMdChecksumFull}}

\vspace{6pt}
Full file names: \texttt{*-\ScanFileTimestamp.txt}. The \texttt{checksums.md} file contains full SHA256 hashes for all scan outputs.

% ----------------------------------------------------------------------------
\dmsection{Verification Chain}

This document establishes a chain of trust for verifying scan integrity:

\vspace{6pt}
\textbf{Chain of Trust:}
\begin{enumerate}[label=\arabic*., leftmargin=0.3in]
    \item \textbf{Digital Signature} $\rightarrow$ This PDF is digitally signed, establishing authenticity
    \item \textbf{checksums.md Hash} $\rightarrow$ This PDF includes the SHA256 hash of \texttt{checksums.md} (Section 7)
    \item \textbf{Scan File Hashes} $\rightarrow$ \texttt{checksums.md} contains full SHA256 hashes of all scan output files
    \item \textbf{File Verification} $\rightarrow$ Any scan file can be verified against \texttt{checksums.md}
\end{enumerate}

\vspace{6pt}
\textbf{Verification Commands:}

\vspace{3pt}
{\small
\begin{verbatim}
# 1. Verify PDF signature (platform-dependent)
# 2. Extract checksums.md hash from PDF Section 7
# 3. Verify checksums.md integrity:
shasum -a 256 .scans/checksums.md

# 4. Verify all scan files against checksums.md:
cd .scans && shasum -a 256 -c checksums.md
\end{verbatim}
}

% ----------------------------------------------------------------------------
\dmsection{Attestation}

This document certifies that:

\begin{enumerate}[label=(\alph*)]
    \item Automated security scans were executed against the subject host at the specified path
    \item Scan results are accurately represented in this document
    \item The Security Verification Toolkit version and commit hash are recorded for traceability
    \item Detailed scan logs are available in the \texttt{.scans/} directory
    \item The verification chain above enables independent validation of all scan outputs
\end{enumerate}

\end{enumerate}

\end{document}
